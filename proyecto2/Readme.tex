
% Especificacion
\documentclass[12pt]{article}

%Paquetes
\usepackage[left=2.5cm,right=2.5cm,top=3cm,bottom=3cm,letterpaper]{geometry}
\usepackage{lmodern}
\usepackage[T1]{fontenc}
\usepackage[utf8]{inputenc}
\usepackage[spanish,activeacute]{babel}
\usepackage{mathtools}
\usepackage{amssymb}
\usepackage{enumerate}
\usepackage{scrextend}
\usepackage{outlines}
\usepackage{graphicx}
\usepackage{titlesec}
\usepackage{enumitem}
\usepackage{alltt}
\usepackage{pmboxdraw}

%Preambulo
\title{Compiladores\\ Proyecto 2: \emph{Análisis Semántico}}
\author{Carlos Acosta \qquad Karla Esquivel \\ Yuan Yuan \qquad Luis Mayo \\ Andrea González}
\date{Facultad de Ciencias UNAM \\ 2017-2}
\setlength\parindent{0pt}

\begin{document}
\maketitle
\tableofcontents
\newpage

\section{Sistema de tipos}
Desde que definimos la gramática para nuestro lenguaje en el \emph{proyecto 1} se habían contemplado los tres tipos que se mencionan en la especificación de este proyecto: tenemos números enteros de tipo \texttt{int}, números con punto flotante de tipo \texttt{float} y valores booleanos \texttt{bool}.
La sintáxis del lenguaje también permitía que se declararan variables con tipo estático. \\
Gracias a que nuestro lenguaje ya cumplía con estos requerimentos, no tuvimos que modificar la sintáxis del lenguaje en estos campos.

\section{Tabla de simbolos}
La \textit{tabla de simbolos} es usada para manejar los tipos de las variables y de los valores que regresen las funciones. Dicha tabla se crea durante el análisis sintáctico usando el patrón \textit{visitor} con la clase \textbf{VisitorCreaTabla} para el descubrimiento de variables, las cuales deben ser inicializadas cuando son declaradas, pero cumplen con la regla de ``declaración antes de uso". \\
Las funciones pueden regresar\footnote{La sentencia \texttt{return} solo puede ser usada dentro de una función, en cualquier otro caso habrá un error de sintáxis} algún valor de tipo \texttt{int}, \texttt{float} o \texttt{bool} o no regresar ninguno (usando \texttt{void}).
\subsection{Implementación}
La tabla de simbolos fue implementada como una estructura de datos en forma de árbol donde la raíz es el alcance principal del programa y cada nuevo alcance que se abra será un hijo para el alcance en el que se abrió.
\subsubsection{métodos principales}
\begin{itemize}
\item \texttt{look\_up(name)}: regresa el nodo que contiene la declaración válida de \emph{name} en el alcance actual. Si no hay declaración, regresa un apuntador nulo.
\item \texttt{insert(name, simbol)}: ingresa \emph{name} en la tabla de simbolos bajo el alcance actual, \emph{simbol} contiene los atributos obtenidos en la declaración de \emph{name}.
\item \texttt{open\_scope()}: abre un nuevo alcance para la tabla de símbolos, es decir, crea un nodo hijo donde se ingresarán nuevos simbolos.
\item \texttt{close\_scope()}: cierra el alcance más reciente, revirtiendo las referencias a simbolos hacia el alcance externo, es decir, a su nodo padre.
\item \texttt{declared\_locally(name)}: verifica que name esté declarado en el alcance actual y regresa \texttt{true} si lo está, en caso contrario regresa \texttt{false}.
\end{itemize}
\section{Verificación de tipos}
Al igual que en la creación de la \textit{tabla de símbolos}, se utilizó el patrón \textit{visitor} con la clase \textbf{VisitorVerificaTipos}. Para auxiliarnos en la verificación de tipos se hace uso de la \textit{tabla de simbolos} creada anteriormente, de manera que se pueda extraer el tipo de cada identificador (ya sea de alguna variable o función) utilizado en el código de entrada.

\section{Análisis dependiente del contexto}
Además de la verificación de tipos, el análisis semántico de este compilador está implementado de modo que para que un programa sea válido para nuestro lenguaje debe cumplir los siguientes requisitos.
\begin{itemize}
\item Solo se puede llamar a funciones que sean definidas previamente.
\item El número de parámetros que recibe la llamada de una función debe ser el mismo que el número de parámetros que se definieron en la firma de dicha función.
\item Los tipos de los parámetros que recibe la llamada de una función deben ser los mismos que los definidos en la firma de la función.
\item En el caso de una asignación, el valor de retorno de la función llamada debe ser del mismo tipo de la variable a la que se está asignando. 
\end{itemize}
\section{Instrucciones para compilar y ejecutar el proyecto}\label{sec:s}

El código fuente del proyecto, se encuentra dentro del directorio \texttt{src/}. Debe asegurarse
que tenga esta forma mínima para su correcto funcionamiento:
\begin{verbatim}
    src/
    |-- lexer.lex
    |-- Makefile
    |-- nodo.cpp
    |-- nodo.h
    |-- parser.y
    |-- tabla.cpp
    |-- tabla.h
    |-- visitor.cpp
    |__ visitor.h

\end{verbatim}

\subsection{Compilación del proyecto}

En cualquier sistema operativo basado en Unix, con \texttt{g++}, \texttt{flex} y \texttt{bison} preinstalados, basta con ejecutar \textbf{make} en el directorio \texttt{src/} desde línea de comandos.

\begin{verbatim}
    [user@host src]$ make
\end{verbatim}
Dicha orden producirá un binario ejecutable de nombre \textit{\textbf{kyc-ip}} listo para recibir código fuente de nuestro lenguaje.\\
En caso de existir un problema para generar dicho ejecutable, recomendamos volver a generar el código de los analizadores sintácticos y volver a intentar construir el ejecutable con las siguientes instrucciones:
\begin{verbatim}
    [user@host src]$ make flex
    [user@host src]$ make bison
    [user@host src]$ make
\end{verbatim}

\subsection{Ejecución del analizador} 

Una vez obtenido el binario, basta ejecutar:
\begin{verbatim}
    [user@host src]$ ./kyc-ip <archivo>
\end{verbatim}
Donde \texttt{<archivo>} es el archivo de texto claro con el código fuente correspondiente a nuestro lenguaje de programación.

\subsection{Ejemplo de un programa no válido}
\begin{verbatim}
fun cmp (int a, int b) int:
    int val = 0;
    cond (a > b):
        val = 1 and true;
    | (a < b):
        val = -1 and false;
    | default:
        val = 0 * 1.23;
    ~cond
    return val;
~fun

fun gungi () void:
    int a = 100;
    int b = a/2;
    int c = 1;
    bool d = false;
    while (c > 0):
        a = (a*b)/0.2;
        b = b*b or d;
        c = cmp(a,b);
    ~while
~fun
\end{verbatim}

Ejecutar este código resultará en los siguientes errores:

\begin{verbatim}
Error de semántica: And de int con bool no válido
Error de semántica: Se está tratando de asignar tipo bool a variable 'val' 
de tipo int
Error de semántica: And de int con bool no válido
Error de semántica: Se está tratando de asignar tipo bool a variable 'val' 
de tipo int
Error de semántica: Multiplicación de int con float no válida
Error de semántica: Se está tratando de asignar tipo float a variable 'val' 
de tipo int
Error de semántica: División de int con float no válida
Error de semántica: Se está tratando de asignar tipo float a variable 'a' de 
tipo int
Error de semántica: Or de int con bool no válido
Error de semántica: Se está tratando de asignar tipo bool a variable 'b' de 
tipo int
\end{verbatim}

\subsection{Ejemplo de un programa válido}
\begin{verbatim}
fun cmp (int a, int b) int:
    int val = 0;
    cond (a > b):
        val = 1;
    | (a < b):
        val = -1;
    | default:
        val = 0;
    ~cond
    return val;
~fun

fun gungi () void:
    int a = 100;
    int b = a/2;
    int c = 1;
    while (c > 0):
        a = a*b;
        b = b*b;
        c = cmp(a,b);
    ~while
~fun
\end{verbatim}

Por otro lado al ejecutar este código el proceso se completará exitosamente sin lanzar ningún error. En ambos casos ya que no habrá error alguno en el análisis sintáctico, se creará el archivo \texttt{archivo.asa} con el árbol de sintáxis abstracto.

\end{document}
