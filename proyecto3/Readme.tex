%Especificacion
\documentclass[12pt]{article}

%Paquetes
\usepackage[left=2.5cm,right=2.5cm,top=3cm,bottom=3cm,letterpaper]{geometry}
\usepackage{lmodern}
\usepackage[T1]{fontenc}
\usepackage[utf8]{inputenc}
\usepackage[spanish,activeacute]{babel}
\usepackage{mathtools}
\usepackage{amssymb}
\usepackage{enumerate}
\usepackage{scrextend}
\usepackage{outlines}
\usepackage{graphicx}
\usepackage{titlesec}
\usepackage{enumitem}
\usepackage{alltt}
\usepackage{pmboxdraw}

%Preambulo
\title{Compiladores\\ Proyecto 3: Generación de Código}
\author{Carlos Acosta \qquad Karla Esquivel \\ Yuan Yuan \qquad Luis Mayo \\ Andrea González}
\date{Facultad de Ciencias UNAM \\ 2017-2}
\setlength\parindent{0pt}

\begin{document}
\maketitle
\section{Preliminares}
Para la realización de este proyecto, hicimos una reducción de nuestra gramática original planteada
desde el proyecto 1, para generar código exclusivamente de expresiones aritméticas, álgebra booleana y asignaciones de variables. Asimismo, conservamos el operador de secuencialidad para no perder la oportunidad de manejar las variables al menos en el primer alcance.
\subsection{La gramática reducida}
Sea $G$ la gramática de nuestro lenguaje de programación \textit{KYC-IP}. Definiremos $G$, como la 4-tupla: $G = (N,T,P,S)$, con $N,T,P$ conjuntos y $S \in N$ el símbolo de la producción inicial.
A continuación, se muestran los conjuntos que la conforman:\\ \\
\texttt{ 
N = \{S, Prog, Prog', Inst, Expr, Bexp, Bterm, Beq, Bcom, Expr', Term, Factor, Bool, Num, Decimal, Entero, Digito, Asig, Asig', Easig, Id, pos, Carac, Letra, Tipo\}
}\\

\texttt{
T = \{;, +, -, /, *, =, \_, ==, >, <, >=, <=, and, or, not, (, ), int, float, bool, 0, 1,..., 9, a, ..., z, A, ..., Z\}  \footnote{En la gramática se utiliza ``!'' para representar a | con el propósito de no confundir el símbolo terminal ``|'' con la separación de las ramificaciones en las producciones.}
}\\

\texttt{
P = \{
\begin{addmargin}[2.5em]{0em}
    S $\to$ Prog\\
    Prog $\to$ Prog Prog' | Prog' \\
    Prog' $\to$ Inst; \\
    Inst $\to$ Expr | Asig\\    
    Expr $\to$ Bexp \\ 
    Bexp $\to$ Bexp or Bterm | Bterm \\ 
    Bterm $\to$ Bterm and Beq | Beq \\ 
    Beq $\to$ Beq == Bcomp | Beq != Bcomp | Bcomp \\ 
    Bcomp $\to$ Bcomp < Expr' | Bcomp > Expr' | Bcomp <= Expr' | Bcomp >= Expr' | Expr' \\ 
    Expr' $\to$ Expr' + Term | Expr' - Term | Term \\
    Term $\to$ Term * Factor | Term / Factor | Factor \\    
    Factor $\to$ Id | Num | (Expr) | - Factor | not Factor | Bool \\
    Bool $\to$ true | false \\
    Num $\to$ Entero Decimal \\
    Decimal $\to$ . Entero | $\varepsilon$ \\
    Entero $\to$ Digito | Digito Entero \\
    Digito $\to$ 0 | 1 | 2 | ... | 9 \\
    Asig $\to$  Easig | Asig' \\
    Asig' $\to$ Tipo Easig \\
    Easig $\to$ Id = Expr | Id = Fun \\
    Id  $\to$ Letra Pos | Letra \\
    Pos  $\to$ Pos Carac | Carac \\
    Carac $\to$ Letra | \_ | Digito \\
    Letra $\to$ a | b | ... | z | A | B | ... | Z \\
    Tipo $\to$ int | float | bool \\
\end{addmargin}
\}
}

\section{Compilación y ejecución del proyecto}
El código fuente del proyecto, se encuentra dentro del directorio \texttt{src/}. Debe asegurarse
que tenga esta forma mínima para su correcto funcionamiento:
\begin{verbatim}
    src/
    |-- ejemplo.kyc
    |-- lexer.lex
    |-- Makefile
    |-- nodo.cpp
    |-- nodo.h
    |-- tabla.cpp
    |-- tabla.h
    |-- visitor.cpp
    |-- visitor.h
    |__ parser.y

\end{verbatim}

\subsection{Compilación del proyecto}

En cualquier sistema operativo basado en Unix, con \texttt{g++}, \texttt{flex} y \texttt{bison} preinstalados, basta con ejecutar \textbf{make} en el directorio \texttt{src/} desde línea de comandos.

\begin{verbatim}
    [user@host src]$ make
\end{verbatim}

Dicha orden producirá un binario ejecutable de nombre \textit{\textbf{kyc-ip}} listo para recibir código fuente de nuestro lenguaje.\\

En caso de existir un problema para generar dicho ejecutable, recomendamos volver a generar el código de los analizadores sintácticos y volver a intentar construir el ejecutable con las siguientes instrucciones:

\begin{verbatim}
    [user@host src]$ make flex
    [user@host src]$ make bison
    [user@host src]$ make
\end{verbatim}


\subsection{Ejecución del analizador} 

Una vez obtenido el binario, basta:
\begin{verbatim}
    [user@host src]$ ./kyc-ip <archivo>
\end{verbatim}
Donde \texttt{<archivo>} es el archivo de texto claro con el código fuente correspondiente a nuestro lenguaje de programación.\\

Además hemos incluído una bandera en el ejecutable que permite generar archivos intermedios
como son el \textit{árbol de sintáxis abstracta} y la \textit{tabla de símbolos}. Para obenterlos
basta agregar a la orden de ejecución lo siguiente:
\begin{verbatim}
    [user@host src]$ ./kyc-ip <archivo> -a
\end{verbatim}
\newpage
\section{Resultado}
Al término de la ejecución podremos encontrar nuevos archivos dentro de la carpeta de recursos del
proyecto.
\begin{verbatim}
    src/
    |-- <archivo>.codigo
    ...
\end{verbatim}
Si hemos elegido omitir la bandera de archivos intermedios. En otro caso, además del código ensamblador encontraremos los siguientes archivos.
\begin{verbatim}
    src/
    |-- <archivo>.codigo
    |-- <archivo>.asa
    |-- <archivo>.tds
    ...
\end{verbatim}
\subsection{Ejemplo}
Por ejemplo para la siguiente \textbf{entrada}:
\begin{verbatim}
int w = 10 * 10 * 10;
float f = ( 56.0 / 89.098 );
bool d = true;
f = f + 10.0 * (10.0 + 11.0) / 76.7788 - 24.95;
d = d and d;


\end{verbatim}
Obtendremos el siguiente \textbf{código objeto}:
\begin{verbatim}
        .data
w:       .quad   0
f:       .float  0.0
d:       .quad   0
        .text
        .globl _start
_start:
movq 1000 %r0   #movq S, D D ← S   LOAD
movq %r0 w      #movq S, D D ← S   SAVE
movss 0.628521 %xmm0    #movss S, D D ← S   LOAD
movss %xmm0 f   #movss S, D D ← S   SAVE
movq 1 %r0      #movq S, D D ← S   LOAD
movq %r0 d      #movq S, D D ← S   SAVE
movss f %xmm1   #movss S, D D ← S   LOAD
addss $2.735130 %xmm1   #addss S, D D ← D + S
subss $24.95 %xmm1      #subss S, D D ← D - S
movss %xmm1 f   #movss S, D D ← S   SAVE
movq 0 %r2      #movq S, D D ← S   LOAD
movq 0 %r1      #movq S, D D ← S   LOAD
andq %r1 %r2    #andq S, D D ← D and S
movq %r2 d      #movq S, D D ← S   SAVE


\end{verbatim}

La siguiente \textbf{tabla de símbolos}:
\begin{verbatim}
{bool: d, int: w, float: f}


\end{verbatim}
Y el siguiente \textbf{ASA}:
\begin{verbatim}
(seq)
├─›(seq)
│  ├─›(seq)
│  │  ├─›(seq)
│  │  │  ├─›(seq)
│  │  │  │  └─›(int)
│  │  │  │     └─›(=)
│  │  │  │        ├─›(w)
│  │  │  │        └─›(*)
│  │  │  │           ├─›(*)
│  │  │  │           │  ├─›(10)
│  │  │  │           │  └─›(10)
│  │  │  │           └─›(10)
│  │  │  └─›(seq)
│  │  │     └─›(float)
│  │  │        └─›(=)
│  │  │           ├─›(f)
│  │  │           └─›(/)
│  │  │              ├─›(56.0)
│  │  │              └─›(89.098)
│  │  └─›(seq)
│  │     └─›(bool)
│  │        └─›(=)
│  │           ├─›(d)
│  │           └─›(true)
│  └─›(seq)
│     └─›(=)
│        ├─›(f)
│        └─›(-)
│           ├─›(+)
│           │  ├─›(f)
│           │  └─›(/)
│           │     ├─›(*)
│           │     │  ├─›(10.0)
│           │     │  └─›(+)
│           │     │     ├─›(10.0)
│           │     │     └─›(11.0)
│           │     └─›(76.7788)
│           └─›(24.95)
└─›(seq)
   └─›(=)
      ├─›(d)
      └─›(and)
         ├─›(d)
         └─›(d)

\end{verbatim}


\end{document}
