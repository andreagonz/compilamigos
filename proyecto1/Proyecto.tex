%Especificacion
\documentclass[12pt]{article}

%Paquetes
\usepackage[left=2cm,right=2cm,top=3cm,bottom=3cm,letterpaper]{geometry}
\usepackage{lmodern}
\usepackage[T1]{fontenc}
\usepackage[utf8]{inputenc}
\usepackage[spanish,activeacute]{babel}
\usepackage{mathtools}
\usepackage{amssymb}
\usepackage{enumerate}
\usepackage{scrextend}
%\usepackage{tabularx}
%\usepackage{wasysym}
\usepackage{graphicx}
%\graphicspath { {media/} }
%\usepackage{pifont}
%Preambulo
\title{Compiladores\\ Proyecto 1}
\author{Carlos Acosta \qquad Karla Esquivel \\ Yuan Yuan \qquad Luis Mayo \\ Andrea González}
\date{Facultad de Ciencias UNAM \\ 2017-2}
\setlength\parindent{0pt}

\begin{document}
\maketitle
\section*{Gramática}
\subsubsection*{1. Definición de la gramática}

Sea $G$ nuestra gramática de expresiones aritméticas. Definiremos $G$, como la 4-tupla: $G = (N,T,P,S)$, con $N,T,P$ conjuntos y $S$ el símbolo de la producción inicial.
A continuación, se muestran los conjuntos que la conforman:\\ \\
\texttt{ 
N = \{Expr, Expr', Asig, Term, Term', Factor, Num, Entero, Decimal, Digito, Var \}
}\\
\texttt{
T = \{., :, ;, +, -, /, *, =, \_, $\sim$, var, cond, !, 0, 1,..., 9, a, ..., z, A, ..., Z\}
}\\
\texttt{
P = \{
\begin{addmargin}[2.5em]{0em}
    S $\to$ Fprog\\
    Fprog $\to$ Fprog Fprog' | Fprog' \\
    Fprog' $\to$ Asig'; | Fundef \\
    Prog $\to$ Prog Prog' | Prog' \\
    Prog' $\to$ Cond | While | Inst; | Fun; \\
    Inst $\to$ Expr | Asig\\
    Fundef $\to$ fun Id (Fdparams) Tipo : Prog $\sim$fun\\
    While $\to$ while Expr : Prog $\sim$while\\    
    Fun $\to$ Id (Fparams)\\
    Fdparams $\to$ Fdparams, Tipo Id | Tipo Id \\
    Fparams $\to$ Fparams, Param | Param  \\
    Param $\to$ Expr \\
    Cond $\to$ cond Expr : Sig $\sim$cond \\
    Sig $\to$ Prog ! Expr : Sig | Prog ! default : Prog | Prog \\
    Expr $\to$ Bexp \\ 
    Bexp $\to$ Bexp or Bterm | Bterm \\ 
    Bterm $\to$ Bterm and Beq | Beq \\ 
    Beq $\to$ Beq == Bcomp | Beq != Bcomp | Bcomp \\ 
    Bcomp $\to$ Bcomp < Expr' | Bcomp > Expr' | Bcomp <= Expr' | Bcomp >= Expr' | Expr' \\ 
    Expr' $\to$ Expr' + Term | Expr' - Term | Term \\
    Term $\to$ Term * Factor | Term / Factor | Factor \\    
    Factor $\to$ Id | Num | (Expr) | - Factor | not Factor | Bool \\
    Bool $\to$ true | false \\
    Num $\to$ Entero Decimal \\
    Decimal $\to$ . Entero | $\varepsilon$ \\
    Entero $\to$ Digito | Digito Entero \\
    Digito $\to$ 0 | 1 | 2 | ... | 9 \\
    Asig $\to$  Easig | Asig' \\
    Asig' $\to$ Tipo Easig \\
    Easig $\to$ Id = Expr | Id = Fun \\
    Id  $\to$ Letra Pos | Letra \\
    Pos  $\to$ Pos Carac | Carac \\
    Carac $\to$ Letra | \_ | Digito \\
    Letra $\to$ a | b | ... | z | A | B | ... | Z \\
    Tipo $\to$ int | float | bool \\
\end{addmargin}
\}
}


\subsubsection*{Identificar las categorías sintácticas y definir los tokens}

\section{Implementación}
\subsubsection*{2. Implementación del analizador sintáctico \textit{Bottom-Up}}
\end{document}