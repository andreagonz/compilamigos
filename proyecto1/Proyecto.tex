%Especificacion
\documentclass[12pt]{article}

%Paquetes
\usepackage[left=2.5cm,right=2.5cm,top=3cm,bottom=3cm,letterpaper]{geometry}
\usepackage{lmodern}
\usepackage[T1]{fontenc}
\usepackage[utf8]{inputenc}
\usepackage[spanish,activeacute]{babel}
\usepackage{mathtools}
\usepackage{amssymb}
\usepackage{enumerate}
\usepackage{scrextend}
\usepackage{outlines}
\usepackage{graphicx}
\usepackage{titlesec}
\usepackage{enumitem}
\usepackage{alltt}
\usepackage{pmboxdraw}

%Preambulo
\title{Compiladores\\ Proyecto 1: kyc-ip}
\author{Carlos Acosta \qquad Karla Esquivel \\ Yuan Yuan \qquad Luis Mayo \\ Andrea González}
\date{Facultad de Ciencias UNAM \\ 2017-2}
\setlength\parindent{0pt}

\begin{document}
\maketitle
\tableofcontents
\newpage

\section{Preliminares}
\subsection{El lenguaje y la gramática}
Definimos la gramática de nuestro lenguaje con mucha inspiración en los lenguajes que ya conocemos, tratando de construirlo con elementos de la sintáxis de éstos, pero considerando las propuestas del equipo también. Idealmente buscamos tomar los aspectos que visual y programáticamente nos agradan en un lenguaje de programación, por supuesto ajustandolo a las funcionalidades que se encomendaban en los lineamientos del proyecto.\\

La base de la gramática de nuestro lenguaje, es la gramática para expresiones aritméticas trabajada en el salón de clases, extendida para secuencias de instrucciones y bloques de estructuras de control y de funciones. Específicamente, ciclos \textit{while}, condicionales \textit{cond} con posibilidad de múltiples sentencias, y funciones. La grámatica del analizador sintáctico de descenso recursivo de la tarea fue de gran ayuda en el diseño. Este proceso nos llevó a pensar en cuáles debían ser las funcionalidades mínimas de nuestro lenguaje. \\

Cabe mencionar que una de las decisiones que nos resultó difícil tomar fue cómo distinguir el final de las estructuras de control y funciones, debido a que no estaba en nuestras intenciones utilizar corchetes. Así que uno de los tokens importantes del lenguaje resultante es para terminar las estructuras de control y las funciones; tomando inspiración de lenguajes que determinan el final de un bloque con un un lexema similar al del inicio del bloque como en el lenguaje de \textit{bash}, terminamos eligiendo el símbolo `$\sim$'\; junto al nombre de la estructura para indicar la terminación de ésta. \\

Al haber incluído, estructuras de control, que por lo general se rigen por expresiones lógicas, decidimos incluír expresiones booleanas y operadores propios de ellas, esto con el fin de que sea más explícita la operación de valores lógicos y dividir la mecánica de uso por tipos en el lenguaje.\\ 

Un ejemplo de código fuente en nuestro lenguaje, puede apreciarse en la Sección \ref{sec:s}.
\section{Definición de la gramática formal}

Sea $G$ la gramática de nuestro lenguaje de programación \textit{KYC-IP}. Definiremos $G$, como la 4-tupla: $G = (N,T,P,S)$, con $N,T,P$ conjuntos y $S \in N$ el símbolo de la producción inicial.
A continuación, se muestran los conjuntos que la conforman:\\ \\
\texttt{ 
N = \{S, Fprog, Fprog', Prog, Prog', Inst, Fundef, Fundef', Return, While, Fun, Fdparams, Fparams, Param, Cond, Sig, Expr, Bexp, Bterm, Beq, Bcom, Expr', Term, Factor, Bool, Num, Decimal, Entero, Digito, Asig, Asig', Easig, Id, pos, Carac, Letra, Tipo\}
}\\

\texttt{
T = \{., :, ;, +, -, /, *, =, \_, ==, >, <, >=, <=, and, or, not, (, ), fun, while, cond, $\sim$fun, $\sim$while, $\sim$cond, int, float, bool, void, |, 0, 1,..., 9, a, ..., z, A, ..., Z\}  \footnote{En la gramática se utiliza ``!'' para representar a | con el propósito de no confundir el símbolo terminal ``|'' con la separación de las ramificaciones en las producciones.}
}\\

\texttt{
P = \{
\begin{addmargin}[2.5em]{0em}
    S $\to$ Fprog\\
    Fprog $\to$ Fprog Fprog' | Fprog' \\
    Fprog' $\to$ Asig'; | Fundef \\
    Prog $\to$ Prog Prog' | Prog' \\
    Prog' $\to$ Cond | While | Inst; | Fun; \\
    Inst $\to$ Expr | Asig\\
    Fundef $\to$ fun Id (Fdparams) Fundef' | fun Id ( ) Fundef' \\
    Fundef' $\to$ Tipo : Prog Return; $\sim$fun | void : Prog $\sim$fun \\
    Return $\to$ return Expr | return Fun \\
    While $\to$ while Expr : Prog $\sim$while\\    
    Fun $\to$ Id (Fparams) | Id ( )\\
    Fdparams $\to$ Fdparams, Tipo Id | Tipo Id \\
    Fparams $\to$ Fparams, Param | Param  \\
    Param $\to$ Expr \\
    Cond $\to$ cond Expr : Sig $\sim$cond \\
    Sig $\to$ Prog ! Expr : Sig | Prog ! default : Prog | Prog \\
    Expr $\to$ Bexp \\ 
    Bexp $\to$ Bexp or Bterm | Bterm \\ 
    Bterm $\to$ Bterm and Beq | Beq \\ 
    Beq $\to$ Beq == Bcomp | Beq != Bcomp | Bcomp \\ 
    Bcomp $\to$ Bcomp < Expr' | Bcomp > Expr' | Bcomp <= Expr' | Bcomp >= Expr' | Expr' \\ 
    Expr' $\to$ Expr' + Term | Expr' - Term | Term \\
    Term $\to$ Term * Factor | Term / Factor | Factor \\    
    Factor $\to$ Id | Num | (Expr) | - Factor | not Factor | Bool \\
    Bool $\to$ true | false \\
    Num $\to$ Entero Decimal \\
    Decimal $\to$ . Entero | $\varepsilon$ \\
    Entero $\to$ Digito | Digito Entero \\
    Digito $\to$ 0 | 1 | 2 | ... | 9 \\
    Asig $\to$  Easig | Asig' \\
    Asig' $\to$ Tipo Easig \\
    Easig $\to$ Id = Expr | Id = Fun \\
    Id  $\to$ Letra Pos | Letra \\
    Pos  $\to$ Pos Carac | Carac \\
    Carac $\to$ Letra | \_ | Digito \\
    Letra $\to$ a | b | ... | z | A | B | ... | Z \\
    Tipo $\to$ int | float | bool \\
\end{addmargin}
\}
}

%razomiento para gramática va aquí 
\section{Categorías sintácticas y definición de tokens}
Elegir los tokens que comprenderían el léxico disponible para nuestro lenguaje de programación, consistió en la observación de las expresiones que era capaz de generar nuestra gramática, proceso que no fue mucho más exhaustivo que el reconocimiento de los símbolos terminales y las cadenas finales que podían producir en combinación. Dada la base de las expresiones aritméticas, los números resultaron una consideración obvia, es por ello que elegimos representar tanto enteros como flotantes que, junto con los booleanos, son los tipos que tenemos contemplados hasta el momento para el lenguaje, los operadores
disponibles para cada uno y los elementos de las estructuras de control y las funciones, fueron igualmente contempladas.\\ 

A continuación se describen las categorías sintácticas identificadas en la grámatica y los tokens\footnote{Por convención, los identificadores de los tokens (átomos) están escritos exclusivamente en mayúsculas.} que la componen:\\
\subsection{Valores:}
\begin{itemize}
  \item ENTERO, números enteros.
    \item FLOTANTE, números de punto flotante.
    \item BOOLEANO, define un valor booleano.
      \begin{itemize}
      \item TRUE, valor booleano de verdadero.
      \item FALSE, valor booleano de falso.
      \item ID, identificadores de variables y funciones.
      \end{itemize}
\end{itemize}
\subsection{Operadores:}
\begin{itemize}
  \item \textbf{Operadores binarios:}
  \begin{itemize}
  \item PLUS, suma para expresiones aritméticas.	
  \item MINUS, resta para expresiones aritméticas.
  \item MULT, producto para expresiones aritméticas.
  \item DIV, división para expresiones aritméticas.
  \item EQ, igualdad de dos expresiones.
  \item NEG, desigualdad de dos expresiones.
  \item AND, operador lógico $\land$ para expresiones booleanas.
  \item OR, operador lógico $\lor$ para expresiones booleanas.
  \item GREAT, operador $>$  para expresiones.
  \item LESS, operador $<$ para expresiones.
  \item GREATEQ, operador $\geq$ para expresiones.
  \item LESSEQ, operador $\leq$ para expresiones.
  \item ASIG, asignación de un valor a una variable.
  \end{itemize}
\item \textbf{Operadores unarios:}
  \begin{itemize}
  \item NOT, negación de un valor.
  \end{itemize}
\end{itemize}
\subsection{Estructuras de control:}
\begin{itemize}
 	\item COND, establece el inicio de un bloque de condicionales.
    \item ENDCOND, establece el final de un bloque de condicionales.
    \item DEFAULT, caso por defecto del bloque de condicionales, en caso de no cumplir con ninguna condición dada.
    \item WHILE, establece el inicio de un bloque de ciclo while.
    \item ENDWHILE, establece el final de un bloque de ciclo while.
    \item FUN, establece el inicio de un bloque de función.
    \item ENDFUN, establece el final de un bloque de función.
\end{itemize}
    \subsection{Conectores:}
    \begin{itemize}
 	\item LPAR, paréntesis izquierdo.
    \item RPAR, paréntesis derecho.
    \item SEMIC, punto y coma para señalar el final de cada sentencia.
    \item DOTDOT, dos puntos para señalar el inicio de un bloque.
    \item COMMA, coma para separar los parametros de una función y añadir legibilidad.
    \item PIPE, señala la declaración de un caso en el bloque de condicionales.
\end{itemize}


\section{Implementación del analizador léxico}
Se usó el generador de analizadores léxicos \textit{Flex} en su versión 2.6.3 para C++. \\
Para más información sobre la implementación del analizador léxico con \textit{Flex} se puede ver el archivo \texttt{lexer.lex} del directorio \texttt{src}.\\
\section{Implementación del analizador sintáctico \textit{Bottom-Up}}
Se utilizó la versión 3.0.4 del generador de analizadores sintácticos \textit{GNU Bison} para C++. \\
Dicha implementación está en el archivo \texttt{parser.y} del directorio \texttt{src/}



\section{Instrucciones para ejecutar y compilar el proyecto}\label{sec:s}

El código fuente del proyecto, se encuentra dentro del directorio \texttt{src/}. Debe asegurarse
que tenga esta forma mínima para su correcto funcionamiento:
\begin{verbatim}
    src/
    |-- ejemplo.kyc
    |-- lexer.lex
    |-- Makefile
    |-- nodo.cpp
    |-- nodo.h
    |__ parser.y

\end{verbatim}

\subsection{Compilación del proyecto}

En cualquier sistema operativo basado en Unix, con \texttt{g++}, \texttt{flex} y \texttt{bison} preinstalados, basta con ejecutar \textbf{make} en el directorio \texttt{src/} desde línea de comandos.

\begin{verbatim}
    [user@host src]$ make
\end{verbatim}

Dicha orden producirá un binario ejecutable de nombre \textit{\textbf{kyc-ip}} listo para recibir código fuente de nuestro lenguaje.\\

En caso de existir un problema para generar dicho ejecutable, recomendamos volver a generar el código de los analizadores sintácticos y volver a intentar construir el ejecutable con las siguientes instrucciones:

\begin{verbatim}
    [user@host src]$ make flex
    [user@host src]$ make bison
    [user@host src]$ make
\end{verbatim}


\subsection{Ejecución del analizador} 

Una vez obtenido el binario, basta:
\begin{verbatim}
    [user@host src]$ ./kyc-ip <archivo>
\end{verbatim}
Donde \texttt{<archivo>} es el archivo de texto claro con el código fuente correspondiente a nuestro lenguaje de programación.



\subsubsection*{Ejemplo de entrada\footnote{El archivo de entrada que contiene este ejemplo se encuentra ubicado en la raíz del proyecto con el nombre de \textbf{\textit{ejemplo.kyc}}. }:}
\begin{verbatim}
fun hola ( int angel ) int : 
    angel = ( angel + 56 / 89.098 );
    bool adan = true or false and true;
    bool armando = false or var1;
    cond adan and armando:
         float c = 1 + 2 / (3.20 * 92);
         | (adan == armando) or (angel > 98.98):
               int x = 98.9;
               9 + 9;
         | default:
               x = f(adan, armando or true, angel + 98.09);            
    ~cond
    while true:
          int x = f2(angel, angel * (0.982 + 2.2) / -89);            
    ~while
    return 0;
~fun

fun adios() void :
    9 + 9;
    f(x);
~fun

int x = f(a, b, c + d);
bool b = false;
\end{verbatim}

\subsection{Salida}
Por cada ejecución se realizará los análisis tanto léxico como sintáctico, de encontrarse algún error en el código fuente, podrá observarse en la salida de la terminal un error señalando la línea y la columna responsables del primer error encontrado que terminó con la ejecución. Si ambos análisis tuvieron éxito, podrá encontrarse dentro de la carpeta raíz del proyecto un archivo de texto (\texttt{<entrada>.asa}) con el árbol de sintáxis abstracta representativo del código fuente de entrada.

\subsubsection*{Ejemplo de ASA}
Dada la siguiente entrada,

\begin{verbatim}
fun hola ( int angel ) int : 
    angel = ( angel + 56 / 89.098 );
    while(true):
       hola(99999999999);
    ~while
    return 0;
~fun
\end{verbatim}
se obtendría el siguiente árbol.
\begin{verbatim}
(fundef)
├─›(hola)
├─›(int)
│  └─›(angel)
└─›(cuerpo)
   ├─›(int)
   ├─›(seq)
   │  ├─›(seq)
   │  │  └─›(=)
   │  │     ├─›(angel)
   │  │     └─›(+)
   │  │        ├─›(angel)
   │  │        └─›(/)
   │  │           ├─›(56)
   │  │           └─›(89.098)
   │  └─›(while)
   │     ├─›(true)
   │     └─›(seq)
   │        └─›(funcall)
   │           ├─›(hola)
   │           └─›(99999999999)
   └─›(return)
      └─›(0)

\end{verbatim}
\end{document}
