%Especificacion
\documentclass[12pt]{article}

%Paquetes
\usepackage[left=2cm,right=2cm,top=3cm,bottom=3cm,letterpaper]{geometry}
\usepackage{lmodern}
\usepackage[T1]{fontenc}
\usepackage[utf8]{inputenc}
\usepackage[spanish,activeacute]{babel}
\usepackage{mathtools}
\usepackage{amssymb}
\usepackage{enumerate}
%\usepackage{tabularx}
%\usepackage{wasysym}
\usepackage{graphicx}
%\graphicspath { {media/} }
%\usepackage{pifont}
%Preambulo
\title{Compiladores\\ Proyecto 1}
\author{Carlos Acosta \qquad Karla Esquivel \\ Yuan Yuan \qquad Luis Mayo \\ Andrea González}
\date{Facultad de Ciencias UNAM \\ 2017-2}
\setlength\parindent{0pt}

\begin{document}
\maketitle
\section*{Gramática}
\subsubsection*{1. Definición de la gramática}

Sea $G$ nuestra gramática de expresiones aritméticas. Definiremos $G$, como la 4-tupla: $G = (N,T,P,S)$, con $N,T,P$ conjuntos y $S$ el símbolo de la producción inicial.
A continuación, se muestran los conjuntos que la conforman:\\

\noindent
\noindent
\texttt{
  \noindent
N = \{Expr, Expr', Asig, Term, Term', Factor, Num, Entero, Decimal, Digito, Var \}\\
T = \{., :, ;, +, -, /, *, =, \_, $\sim$, var, cond, !, 0, 1,..., 9, a, ..., z, A, ..., Z\}\\ \\ 
P = \{}
\begin{quote}
  \texttt{
    S $\to$ Fprog\\
    Fprog $\to$ Fprog Fundef | Fprog Prog | Prog | Fundef\\
    Prog $\to$ Inst; | Fun; | Cond | While | Prog Inst; | Prog Fun; | Prog Cond | Prog While\\
    Inst $\to$ Expr | Asig\\
    Fundef $\to$ fun Fnom (Fdparams) Tipo : Prog $\sim$fun\\
    While $\to$ while Ebool : Prog $\sim$while\\    
    Fun $\to$ Id (Fparams)\\
    Fdparams $\to$ Tipo Id | Fdparams, Tipo Id\\
    Fparams $\to$ Id | Fparams, Id | Fun | Fparams, Fun\\
    Cond $\to$ cond Ebool : Sig $\sim$cond \\
    Sig $\to$ Prog | Prog ! Ebool : Sig | Prog ! default : Prog \\
    Expr  $\to$ Term Expr' | Ebool \\
    Expr' $\to$ + Term Expr' | - Term Expr' | $\varepsilon$ \\
    Term  $\to$ Factor Term' \\
    Term' $\to$ * Factor Term' | / Factor Term'| $\varepsilon$ \\
    Factor $\to$ Num | var | (Expr) | - Expr \\
    Num $\to$ Entero Decimal \\
    Decimal $\to$ . Entero | $\varepsilon$ \\
    Entero $\to$ Digito | Digito Entero \\
    Digito $\to$ 0 | 1 | 2 | ... | 9 \\
    Ebool $\to$ ... \\
    Asig $\to$ Tipo Id = Expr | Id = Expr \\
    Id  $\to$ Letra Pos | Letra \\
    Pos  $\to$ Pos Carac | Carac \\
    Carac $\to$ Letra | \_ | Digito
    Letra $\to$ a | b | ... | z | A | B | ... | Z \\
    Tipo $\to$ int | float | bool \\
    \}
}
 
\end{quote}

\subsubsection*{Identificar las categorías sintácticas y definir los tokens}

\section{Implementación}
\subsubsection*{2. Implementación del analizador sintáctico \textit{Bottom-Up}}
\end{document}